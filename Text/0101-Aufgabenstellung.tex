\chapter{Aufgabenstellung}
\label{sec: Aufgabenstellung}

\section{Laborübung 1 - Sensorkalibrierung}

Aufgabe der Laborübung ist es geeignete Messungen mit der IMU durchzuführen um diese aufgrund der gemessenen Signale kalibrieren zu können. Beim Kalibriervorgang sollten sowohl der Offsetfehler, als auch der Skalierungsfehler der einzelnen Sensorachsen korrigiert werden. Außerdem sollten zusammengehörige Sensordaten auf einen Referenzwert normiert werden (z.B. Beschleunigungssensoren in x-,y- und z-Achse auf den Wert $9,81\frac{m}{s^2}$), um ein korrektes Messergebnis gewährleisten zu können. Die Sensoren sollen auch bezüglich ihrer Varianz beurteilt werden.
Um die systematischen Fehler der Sensoren zu korrigieren sollen zunächst eine Reihe von Messungen durchgeführt werden. Alle Messungen sind in Tabelle \ref{tab:Messungen} gelistet.

\begin{table}[h]
	\centering
	\begin{tabular}{|p{2cm}|p{10.6cm}|p{2cm}|}
		\hline
		\rowcolor{lightgray}\textbf{Messung} & \textbf{Durchführung} & \textbf{Dauer} \\
		\hline
		1	&Messung der Erdbeschleunigung in positive x-Richtung			& 30 s\\
		\hline
		2	&Messung der Erdbeschleunigung in positive y-Richtung			& 30 s\\
		\hline
		3	&Messung der Erdbeschleunigung in positive z-Richtung			& 30 s\\
		\hline
		4	&Messung der Erdbeschleunigung in negative x-Richtung			& 30 s\\
		\hline
		5	&Messung der Erdbeschleunigung in negative y-Richtung			& 30 s\\
		\hline
		6	&Messung der Erdbeschleunigung in negative z-Richtung			& 30 s\\
		\hline
		7	&Messung des Magnetfeldes in x-, y- und z-Richtung – Sensor wird dabei um alle Achsen zufällig rotiert	& 60 s\\
		\hline
		8	&Messung der Drehraten um die x-, y- und z-Achse – Sensor wird dabei nicht bewegt!	& 30 s\\
		\hline
	\end{tabular}
	\caption{Messungen zur Sensorkalibrierung}
	\label{tab:Messungen}
\end{table}

\subsection{Abgleich Beschleunigungssensor}
Für die Kalibrierung des Beschleunigungssensors werden die Messungen 1-6 aus Tabelle \ref{tab:Messungen} herangezogen. Als Referenzwert für die Kalibrierung wird die Erdbeschleunigung $g=9,81\frac{m}{s^2}$) verwendet, welche stets konstant ist und bei entsprechender Orientierung der IMU immer nur entlang einer der drei Koordinatenachsen wirkt. Aus jeweils einem Paar von Messungen (z.B. aus den Messungen 1 und 4) können a\textsubscript{min} und a\textsubscript{max} berechnet werden, wobei a\textsubscript{min} und a\textsubscript{max} jeweils aus dem Mittelwert der Messfolge jener Koordinatenachse sind. Anschließend kann der durchschnittliche Endausschlag bei voller Erdbeschleunigung a\textsubscript{peak} ermittelt werden.
\begin{align}
	a_{peak}=\frac{\vert{a_{min}}\vert+\vert{a_{max}}\vert}{2}
	\label{eq: a_peak}
\end{align}
Über den Referenzwert g kann anschließend der Skalierungsfaktor für die Normierung (s\textsubscript{norm}) berechnet werden.
\begin{align}
	s_{norm}=\frac{g}{a_{peak}}
	\label{eq: s_norm}
\end{align}
Der entsprechende normierte Offsetabgleich lässt sich wie folgt berechnen berechnen.
\begin{align}
	a_{off}=(a_{max}-a_{peak})\cdot{s_{norm}}
	\label{eq: a_off}
\end{align}
Der korrigierte Wert a\textsubscript{ist} kann aus dem Messwert a\textsubscript{m} berechnet werden.
\begin{align}
	a_{ist}=a_m\cdot{s_{norm}}-a_{off}
	\label{eq: a_ist}
\end{align}
Diese Vorgehensweise ist für jede Achse durchzuführen.

\subsection{Abgleich Magnetometer}
Für die Kalibrierung des Magnetometers wird Messung 7 aus Tabelle \ref{tab:Messungen} herangezogen. Als Referenzwert für die Kalibrierung kann die mittlere Stärke des Erdmagnetfeldes in Mitteleuropa m\textsubscript{e} verwendet werden. Dies beträgt 48 μT. Aus jeweils der Messfolge der jeweiligen Koordinatenachse kann m\textsubscript{min} und m\textsubscript{max} bestimmt werden, wobei m\textsubscript{min} und m\textsubscript{max} über eine Maximalwertsuche bzw. eine Minimalwertsuche bestimmt wird. Analog zu den für den Beschleunigungssensor durchgeführten Berechnungen lassen sich anschließend auch für das Magnetometer die Kalibrierwerte über die nachfolgenden Formeln berechnen.
\begin{align}
	m_{peak}=\frac{\vert{m_{min}}\vert+\vert{m_{max}}\vert}{2}
	\label{eq: m_peak}
\end{align}
\begin{align}
	m_{norm}=\frac{m_e}{m_{peak}}
	\label{eq: m_norm}
\end{align}
\begin{align}
	m_{off}=(m_{max}-m_{peak})\cdot{s_{norm}}
	\label{eq: m_off}
\end{align}
\begin{align}
	m_{ist}=m_m\cdot{s_{norm}}-m_{off}
	\label{eq: m_ist}
\end{align}

\subsection{Abgleich Gyroskope}
Für die Kalibrierung des Gyroskops wird die Messung 8 aus Tabelle \ref{tab:Messungen} herangezogen. Da keine definierte Drehbewegung um eine Koordinatenachse im Labor durchgeführt werden kann, kann kein Referenzwert für das Gyroskop bestimmt werden. Aus diesem Grund kann die Skalierung des Gyroskops nicht kalibriert werden. Beim Gyroskop kann lediglich ein Offsetabgleich durchgeführt werden. Da sich bei den Messungen 8 der Sensor in Ruhelage befand, muss die Winkelgeschwindigkeit gleich Null sein. Durch Mittelwertbildung über die Messwertfolgen der Winkelgeschwindigkeiten können die Offsetwerte aller Achsen berechnet werden.
\begin{align}
	\omega_{ist}=\omega_m-\omega_{off}
	\label{eq: omega_ist}
\end{align}

\subsection{Temperaturabgleich}
Es wird eine Messreihe der kalibrierten Werte aufgenommen. Dabei wird der Sensor auf eine maximale Temperatur von 80 °C erwärmt. Im Anschluss können alle Einzelsensorwerte über der Temperatur dargestellt werden und eine Ausgleichsfunktion bestimmt werden.

\subsection{Bestimmung der Varianz der einzelnen Sensoren}
Für Bestimmung der Varianz kann die Messung 8 aus Tabelle \ref{tab:Messungen} herangezogen werden.

\section{Laborübung 2 - Kalman Filter}
Es soll ein Kalman-Filter in Matlab implementiert werden:
\begin{enumerate}
	\item	Prädiktor-Schritt: \\
	Bestimmung des Neigungswinkel mit Hilfe des Gyroskops.
	\begin{align}
		\bar{\mu}_{t+1}=A\cdot{\mu_t}+B\cdot{u_{t+1}} \\
		\bar{\Sigma}_{t+1}=A\cdot{\Sigma_t}\cdot{A^T}+\Sigma_u
		\label{eq: Predictor}
	\end{align}
	\item	Korrektur-Schritt: \\
	Die Korrektur des Neigungswinkel wird mit Hilfe des berechneten Winkels der Beschleunigungssensoren durchgeführt.
	\begin{align}
		K_{t+1}=\bar{\Sigma}_{t+1}\cdot{H^T}\cdot{(H\cdot{\bar{\Sigma}_{t+1}}\cdot{H^T}+\Sigma_z)^{-1}} \\
		\mu_{t+1}=\bar{\mu}_{t+1}+K_{t+1}\cdot{(z_{t+1}-H\cdot{\bar{\mu}_{t+1}})} \\
		\Sigma_{t+1}=(I-K_{t+1}\cdot{H})\cdot{\bar{\Sigma}_{t+1}}
		\label{eq: Corrector}
	\end{align}
\end{enumerate}
Die Matrizen für die Drehbewegung um eine Achse können wie folgt eingegeben werden (siehe Skript):

\begin{equation*}
A = 
\begin{bmatrix}
1 & -dt \\
0 & 1   \\
\end{bmatrix}
\end{equation*}
\begin{equation*}
B = 
\begin{bmatrix}
dt \\
0  \\
\end{bmatrix}
\end{equation*}
\begin{equation*}
H = 
\begin{bmatrix}
1 & 0 \\
\end{bmatrix}
\end{equation*}
\begin{equation*}
I = 
\begin{bmatrix}
1 & 0 \\
0 & 1 \\
\end{bmatrix}
\end{equation*}

Um eine Überprüfung der Funktionalität des zu implementierenden Kalman-Filters durchzuführen zu können, sollen IMU-Messungen, wie in Tabelle  \ref{tab:Messungen} angegeben, aufgenommen werden. Es werden zuerst Messungen aufgenommen, in welcher nur rotatorische Bewegungen getätigt wurden (Messung 1 - 3). In Messung 4 soll gezeigt werden, dass störende Querbeschleunigungen kompensiert werden können.

\begin{table}[h]
	\centering
	\begin{tabular}{|p{2cm}|p{10.6cm}|p{2cm}|}
		\hline
		\rowcolor{lightgray}\textbf{Messung} & \textbf{Durchführung} & \textbf{Dauer} \\
		\hline
		1	&Rotation um die x-Achse (±45 °)		& 30 s\\
		\hline
		2	&Rotation um die y-Achse (±45 °)		& 30 s\\
		\hline
		3	&Rotation um die z-Achse (±45 °)		& 30 s\\
		\hline
		4	&Rotation um die x-Achse und gleichzeitiges Bewegen in die positive und negative y-Richtung			& 30 s\\
		\hline
	\end{tabular}
	\caption{Messungen zur Überprüfung des Kalman- Filters}
	\label{tab:Angabe Kalman-Filter}
\end{table}
