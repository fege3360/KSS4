\chapter{Literaturverzeichnis}
\label{sec: Bibliography}
Nachstehend finden Sie zwei Muster-Literaturverzeichnisse, die die wichtigsten Arten von Veröffentlichungen abdecken (reine Internet-Quelle, Norm, Monografie, Aufsatz in einem Sammelband, technische Regel, Zeitschriftenaufsatz). Informationen zu weiteren Publikationstypen finden Sie für das Deutsche in der unten genannten ONR~12658, für das Englische in der unten genannten ISO~690. Beide Regelwerke sind in der Bibliothek des FH-OÖ-Campus Wels einsehbar.

\section{Literaturverzeichnis bei Nutzung von Quellenangaben im Text (PDK) oder in der Fußnote (MEWI)}
\printbibliography[heading=none,env=bibliographyAlpha]
\nocite{*}

\section{Literaturverzeichnis bei Nutzung von Quellenangaben in Form von Zahlen in eckigen Klammern (VTP, MEWI, MKT)}
\printbibliography[heading=none]

\newpage
\subsection*{Erscheinungsbild}
Das Aussehen des Literaturverzeichnisses hängt von der Zitierweise ab, die Sie in Ihrer wissenschaftlichen Arbeit anwenden müssen [siehe Abschnitt \ref{sec: ZitierStil} auf S. \pageref{sec: ZitierStil}].

\subsection*{Gliederung}
Das Literaturverzeichnis kann bei Bedarf in folgende Unterkapitel untergliedert werden:
\begin{itemize}
	\item	Primärliteratur 
	\item	Sekundärliteratur 
	\item	Tertiärliteratur
\end{itemize}

\subsection*{Literaturrecherche}
Auf den Internetseiten der Bibliothek des FH-OÖ-Campus Wels finden Sie eine sehr übersichtliche Liste mit zahlreichen Links zu:
\begin{itemize}
	\item	elektronischen Zeitschriften
	\item	Datenbanken
	\item	zahlreichen Bibliotheken
	\item	Katalogen des Buchhandels
	\item	Patentgesellschaften
\end{itemize}
Die MitarbeiterInnen der Bibliothek unterstützen Sie gerne bei der Literaturrecherche.

